\documentclass{introtosml}
\nth{2}

\newcommand\Ep[1]{E_{p(\vec{x}, y)} \left[ #1 \right]}
\newcommand\fancyb{\beta_{\hat{y}, y}}

\begin{document}

\maketitle

\begin{p}
  \item
    \begin{align*}
      R(hr; q)
      & = \int_\vec{x} \sum_{c=1}^C \sum_{c'=1}^C
          L_{0/1} (c', c) q(c_r = c' | \vec{x}) p(\vec{x}, y = c) d\vec{x} \\
      & = \int_\vec{x} R(h_r | \vec{x}) p(\vec{x}) d\vec{x} \\
      \text{where} R(h_r | \vec{x})
      = \sum_{c=1}^C \sum_{c'=1}^C L_{0/1} (c', c) q(c_r = c' | \vec{x}) p(y = c | \vec{x}) \\
      R(h^*)
      & = \int_\vec{x} R(h^* | \vec{x}) p(\vec{x}) d\vec{x} \\
      \text{where} R(h^* | \vec{x})
      = \sum_{c=1}^C L_{0/1} (h^*(\vec{x}), c) p(y = c | \vec{x}) \\
    \end{align*}

  \item
    \begin{align*}
      \frac{\partial}{\partial \vec{w}} R(\vec{w}) & = 0 \\
      \frac{\partial}{\partial \vec{w}} \Ep{(y - {\vec{w}}^T \vec{x})^2 } & = 0 \\
      \Ep{2 (y - {\vec{w}}^T \vec{x}) (-\vec{x})} & = 0 \\
      \Ep{(y - {\vec{w}}^T \vec{x}) \vec{x}} & = 0 \label{eq:corr0} \\
      A \Ep{(y - {\vec{w}}^T \vec{x}) \vec{x}} & = 0 \\
      \Ep{(y - {\vec{w}}^T \vec{x}) A \vec{x}} & = 0
    \end{align*}

  \item
    \begin{align*}
      \Ep{y - \vec{w}^T \vec{x}} & = 0 \\
      \Ep{(y - \vec{w}^T \vec{x}) E_{p(\vec{x})} \left[ A \vec{x} \right]} & = 0 \\
      \Ep{(y - \vec{w}^T \vec{x}) A \vec{x}}
          - \Ep{(y - \vec{w}^T \vec{x}) E_{p(\vec{x})} \left[ A \vec{x} \right]} & = 0 \\
      \Ep{(y - \vec{w}^T \vec{x}) (A \vec{x} - E_{p(\vec{x})} \left[ A \vec{x} \right])} & = 0
    \end{align*}
    \therefore the correration between any linear function of data and prediction errors is 0.

  \item
    \begin{gather*}
      \begin{aligned}
        \hat{\vec{w}}
        & = \argmin_\vec{w} \sum_{i=1}^N (y_i - \vec{w}^T \vec{x}_i)^2 \\
        & = (X^T X)^{-1} X^T y
      \end{aligned} \\
      \text{Let} C \in \field{R}^{(d+1) \times (d+1)} \text{be a diagonal matrix}
          \st \tilde{X} = XC \\
      \begin{aligned}
        \hat{\tilde{\vec{w}}}
        & = \argmin_\vec{w} \sum_{i=1}^N (y_i - \vec{w}^T \tilde{\vec{x}}_i)^2 \\
        & = (\tilde{X}^T \tilde{X})^{-1} \tilde{X}^T y \\
        & = ((XC)^T XC)^{-1} (XC)^T y \\
        & = (CX^T XC)^{-1} CX^T y \\
        & = C^{-1} (X^T X)^{-1} C^{-1} CX^T y \\
        & = C^{-1} (X^T X)^{-1} X^T y
      \end{aligned} \\
      \begin{aligned}
        \tilde{X}\hat{\tilde{\vec{w}}}
        & = XCC^{-1} (X^T X)^{-1} X^T y \\
        & = X (X^T X)^{-1} X^T y \\
        & = X \hat{\vec{w}} \text{as required}
      \end{aligned}
    \end{gather*}

  \item
    \begin{gather*}
      \begin{aligned}
        \hat{\sigma^2}
        & = \argmax_{\sigma^2} \sum_{i=1}^N \log p(y_i|\vec{x}_i; \vec{w}, \sigma) \\
        & = \argmax_{\sigma^2} - \frac{1}{2 \sigma^2}
            \sum_{i=1}^N (y_i - f(\vec{x}_i; \vec{w}))^2 - N \log \sigma \sqrt{2 \pi} \\
        & = \argmin_{\sigma^2} \frac{1}{\sigma^2}
            \sum_{i=1}^N (y_i - f(\vec{x}_i; \vec{w}))^2 + N \log 2 \pi \sigma^2 \\
        & = \argmin_{\sigma^2} \frac{1}{\sigma^2}
            \sum_{i=1}^N (y_i - f(\vec{x}_i; \vec{w}))^2 + N \log \sigma^2
      \end{aligned} \\
      \begin{aligned}
      \frac{\partial}{\partial \sigma^2} \frac{1}{\sigma^2}
          \sum_{i=1}^N (y_i - f(\vec{x}_i; \vec{w}))^2 + N \log \sigma^2 & = 0 \\
      - \frac{1}{\sigma^4}
          \sum_{i=1}^N (y_i - f(\vec{x}_i; \vec{w}))^2 + \frac{N}{\sigma^2} & = 0 \\
      \sum_{i=1}^N (y_i - f(\vec{x}_i; \vec{w}))^2 - N \sigma^2 & = 0 \\
      \sigma^2 & = \frac{1}{N} \sum_{i=1}^N (y_i - f(\vec{x}_i; \vec{w}))^2 \\
      \therefore \hat{\sigma^2} & = \frac{1}{N} \sum_{i=1}^N (y_i - f(\vec{x}_i; \vec{w}))^2
      \end{aligned}
    \end{gather*}

    The experimental result on a validation dataset showed that there are huge
    gaps of loss, $\sigma^2$, and log-likelihood between a linear model and, a quadratic
    and cubic models.
    The lienar model showed much greater values in terms of loss and $\sigma^2$.
    The others achieved much better log-likelihood values.

    Based on the evaluation on a validation dataset, I select a quadratic model
    as model A.
    The reasons are listed below.
    \begin{itemize}
      \item Computional efficiency for training and prediction \\
            There are fewer times of multiplication compared with a cubic one.
      \item Low complexity \\
            It has one fewer parameters compared with a cubic one.
      \item Acceptably low loss value \\
            While a linear model is more efficient and simpler than a quadratic one,
            it showed too large loss on both training and validation datasets.
            That means a linear one is not expressive enough for the data.
    \end{itemize}

  \item
    \begin{gather*}
      \text{Let} \fancyb = \begin{cases}
        1 & \mif \hat{y} \le y \\
        \alpha & \otherwise
      \end{cases} \\
      \begin{aligned}
        l_\alpha(\hat{y}, y)
        & = \begin{cases}
          (\hat{y} - y)^2 & \mif \hat{y} \le y \\
          \alpha (\hat{y} - y)^2 & \otherwise
        \end{cases} \\
        & = \fancyb (\hat{y} - y)^2
      \end{aligned} \\
      \begin{aligned}
        \frac{\partial}{\partial \vec{w}} L_\alpha
        & = \frac{\partial}{\partial \vec{w}}
            \frac{1}{N} \sum_{i=1}^N l_\alpha(\hat{y}, y) \\
        & = \frac{\partial}{\partial \vec{w}}
            \frac{1}{N} \sum_{i=1}^N \fancyb (\hat{y}_i - y_i)^2 \\
        & = \frac{\partial}{\partial \vec{w}}
            \frac{1}{N} \sum_{i=1}^N \fancyb (\vec{w}^T \phi_i(\vec{x}) - y_i)^2 \\
            & = \frac{1}{N} \sum_{i=1}^N 2 \fancyb (\vec{w}^T \phi_i(\vec{x}) - y_i) \phi_i(\vec{x}) \\
      \end{aligned}
    \end{gather*}

    Here, I also choose a quadratic model as model B with the exactly same reasons as model A.

    While I cannot compare model A and B because I chose the same quadratic one,
    I think it is not reasonable to compare them because they are evaluated
    on different tasks with symmetric and asymmetric loss functions.
    The results of different experiments just explain which model is better
    on which experiment.
\end{p}

\end{document}
